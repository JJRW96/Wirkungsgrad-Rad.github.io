% Options for packages loaded elsewhere
\PassOptionsToPackage{unicode}{hyperref}
\PassOptionsToPackage{hyphens}{url}
\PassOptionsToPackage{dvipsnames,svgnames,x11names}{xcolor}
%
\documentclass[
  letterpaper,
  DIV=11]{scrartcl}

\usepackage{amsmath,amssymb}
\usepackage{iftex}
\ifPDFTeX
  \usepackage[T1]{fontenc}
  \usepackage[utf8]{inputenc}
  \usepackage{textcomp} % provide euro and other symbols
\else % if luatex or xetex
  \usepackage{unicode-math}
  \defaultfontfeatures{Scale=MatchLowercase}
  \defaultfontfeatures[\rmfamily]{Ligatures=TeX,Scale=1}
\fi
\usepackage{lmodern}
\ifPDFTeX\else  
    % xetex/luatex font selection
\fi
% Use upquote if available, for straight quotes in verbatim environments
\IfFileExists{upquote.sty}{\usepackage{upquote}}{}
\IfFileExists{microtype.sty}{% use microtype if available
  \usepackage[]{microtype}
  \UseMicrotypeSet[protrusion]{basicmath} % disable protrusion for tt fonts
}{}
\makeatletter
\@ifundefined{KOMAClassName}{% if non-KOMA class
  \IfFileExists{parskip.sty}{%
    \usepackage{parskip}
  }{% else
    \setlength{\parindent}{0pt}
    \setlength{\parskip}{6pt plus 2pt minus 1pt}}
}{% if KOMA class
  \KOMAoptions{parskip=half}}
\makeatother
\usepackage{xcolor}
\setlength{\emergencystretch}{3em} % prevent overfull lines
\setcounter{secnumdepth}{5}
% Make \paragraph and \subparagraph free-standing
\makeatletter
\ifx\paragraph\undefined\else
  \let\oldparagraph\paragraph
  \renewcommand{\paragraph}{
    \@ifstar
      \xxxParagraphStar
      \xxxParagraphNoStar
  }
  \newcommand{\xxxParagraphStar}[1]{\oldparagraph*{#1}\mbox{}}
  \newcommand{\xxxParagraphNoStar}[1]{\oldparagraph{#1}\mbox{}}
\fi
\ifx\subparagraph\undefined\else
  \let\oldsubparagraph\subparagraph
  \renewcommand{\subparagraph}{
    \@ifstar
      \xxxSubParagraphStar
      \xxxSubParagraphNoStar
  }
  \newcommand{\xxxSubParagraphStar}[1]{\oldsubparagraph*{#1}\mbox{}}
  \newcommand{\xxxSubParagraphNoStar}[1]{\oldsubparagraph{#1}\mbox{}}
\fi
\makeatother


\providecommand{\tightlist}{%
  \setlength{\itemsep}{0pt}\setlength{\parskip}{0pt}}\usepackage{longtable,booktabs,array}
\usepackage{calc} % for calculating minipage widths
% Correct order of tables after \paragraph or \subparagraph
\usepackage{etoolbox}
\makeatletter
\patchcmd\longtable{\par}{\if@noskipsec\mbox{}\fi\par}{}{}
\makeatother
% Allow footnotes in longtable head/foot
\IfFileExists{footnotehyper.sty}{\usepackage{footnotehyper}}{\usepackage{footnote}}
\makesavenoteenv{longtable}
\usepackage{graphicx}
\makeatletter
\def\maxwidth{\ifdim\Gin@nat@width>\linewidth\linewidth\else\Gin@nat@width\fi}
\def\maxheight{\ifdim\Gin@nat@height>\textheight\textheight\else\Gin@nat@height\fi}
\makeatother
% Scale images if necessary, so that they will not overflow the page
% margins by default, and it is still possible to overwrite the defaults
% using explicit options in \includegraphics[width, height, ...]{}
\setkeys{Gin}{width=\maxwidth,height=\maxheight,keepaspectratio}
% Set default figure placement to htbp
\makeatletter
\def\fps@figure{htbp}
\makeatother
% definitions for citeproc citations
\NewDocumentCommand\citeproctext{}{}
\NewDocumentCommand\citeproc{mm}{%
  \begingroup\def\citeproctext{#2}\cite{#1}\endgroup}
\makeatletter
 % allow citations to break across lines
 \let\@cite@ofmt\@firstofone
 % avoid brackets around text for \cite:
 \def\@biblabel#1{}
 \def\@cite#1#2{{#1\if@tempswa , #2\fi}}
\makeatother
\newlength{\cslhangindent}
\setlength{\cslhangindent}{1.5em}
\newlength{\csllabelwidth}
\setlength{\csllabelwidth}{3em}
\newenvironment{CSLReferences}[2] % #1 hanging-indent, #2 entry-spacing
 {\begin{list}{}{%
  \setlength{\itemindent}{0pt}
  \setlength{\leftmargin}{0pt}
  \setlength{\parsep}{0pt}
  % turn on hanging indent if param 1 is 1
  \ifodd #1
   \setlength{\leftmargin}{\cslhangindent}
   \setlength{\itemindent}{-1\cslhangindent}
  \fi
  % set entry spacing
  \setlength{\itemsep}{#2\baselineskip}}}
 {\end{list}}
\usepackage{calc}
\newcommand{\CSLBlock}[1]{\hfill\break\parbox[t]{\linewidth}{\strut\ignorespaces#1\strut}}
\newcommand{\CSLLeftMargin}[1]{\parbox[t]{\csllabelwidth}{\strut#1\strut}}
\newcommand{\CSLRightInline}[1]{\parbox[t]{\linewidth - \csllabelwidth}{\strut#1\strut}}
\newcommand{\CSLIndent}[1]{\hspace{\cslhangindent}#1}

\KOMAoption{captions}{tableheading}
\makeatletter
\@ifpackageloaded{caption}{}{\usepackage{caption}}
\AtBeginDocument{%
\ifdefined\contentsname
  \renewcommand*\contentsname{Inhaltsverzeichnis}
\else
  \newcommand\contentsname{Inhaltsverzeichnis}
\fi
\ifdefined\listfigurename
  \renewcommand*\listfigurename{Abbildungsverzeichnis}
\else
  \newcommand\listfigurename{Abbildungsverzeichnis}
\fi
\ifdefined\listtablename
  \renewcommand*\listtablename{Tabellenverzeichnis}
\else
  \newcommand\listtablename{Tabellenverzeichnis}
\fi
\ifdefined\figurename
  \renewcommand*\figurename{Abbildung}
\else
  \newcommand\figurename{Abbildung}
\fi
\ifdefined\tablename
  \renewcommand*\tablename{Tabelle}
\else
  \newcommand\tablename{Tabelle}
\fi
}
\@ifpackageloaded{float}{}{\usepackage{float}}
\floatstyle{ruled}
\@ifundefined{c@chapter}{\newfloat{codelisting}{h}{lop}}{\newfloat{codelisting}{h}{lop}[chapter]}
\floatname{codelisting}{Listing}
\newcommand*\listoflistings{\listof{codelisting}{Listingverzeichnis}}
\makeatother
\makeatletter
\makeatother
\makeatletter
\@ifpackageloaded{caption}{}{\usepackage{caption}}
\@ifpackageloaded{subcaption}{}{\usepackage{subcaption}}
\makeatother
\ifLuaTeX
\usepackage[bidi=basic]{babel}
\else
\usepackage[bidi=default]{babel}
\fi
\babelprovide[main,import]{ngerman}
% get rid of language-specific shorthands (see #6817):
\let\LanguageShortHands\languageshorthands
\def\languageshorthands#1{}
\ifLuaTeX
  \usepackage{selnolig}  % disable illegal ligatures
\fi
\usepackage{bookmark}

\IfFileExists{xurl.sty}{\usepackage{xurl}}{} % add URL line breaks if available
\urlstyle{same} % disable monospaced font for URLs
\hypersetup{
  pdflang={de},
  colorlinks=true,
  linkcolor={blue},
  filecolor={Maroon},
  citecolor={Blue},
  urlcolor={Blue},
  pdfcreator={LaTeX via pandoc}}

\author{}
\date{}

\begin{document}

\section*{Abkürzungsverzeichnis}\label{abkuxfcrzungsverzeichnis}
\addcontentsline{toc}{section}{Abkürzungsverzeichnis}

\[
\begin{alignat*}{2}
&A_{\text{fast}} \text{ [ml·min}^{-1}\text{]}:&& \text{ Amplitude der schnellen EPOC-Komponente} \\[1ex]
&A_{\text{slow}} \text{ [ml·min}^{-1}\text{]}:&& \text{ Amplitude der langsamen EPOC-Komponente} \\[1ex]
&C_{\text{a}} \text{ [ml O}_2\text{·100ml}^{-1}\text{]}:&& \text{ Arterielle Sauerstoffkonzentration} \\[1ex]
&C_{\text{v}} \text{ [ml O}_2\text{·100ml}^{-1}\text{]}:&& \text{ Venöse Sauerstoffkonzentration} \\[1ex]
&\Delta\text{BLC} \text{ [mmol·l}^{-1}\text{]}: && \text{ Netto-Blutlaktatakkumulation (Post - Pre)} \\[1ex]
&\Delta\text{G} \text{ [kJ]}: && \text{ Änderung der Gibbs'schen freien Energie} \\[1ex]
&\Delta\dot{\text{V}}\text{O}_{2} \text{ [l·min}^{-1}\text{]}: && \text{ Differenz zwischen zwei Sauerstoffvolumenströmen bzw. die Amplitude des Sauerstoffvolumenstroms} \\[1ex]
&\text{E}_{\text{kin,rot}}: && \text{ Rotatorische kinetische Energiekomponente} \\[1ex] 
&\text{E}_{\text{kin,trans}}: && \text{ Translatorische kinetische Energiekomponente} \\[1ex]
&\text{E}_{\text{Tot}} \text{ [kJ]}: && \text{ Gesamte umgesetzte chemische Energie} \\[1ex]
&\text{EPOC}_{\text{fast}} \text{ [l]}:&& \text{ EPOC der schnellen Komponente} \\[1ex]
&\text{EPOC}_{\text{fast}} \text{ [ml·kg}^{-1}\text{]}: && \text{ Auf Körpermasse normierte EPOC der schnellen EPOC} \\[1ex]
&\text{EPOC}_{\text{ges}} \text{ [l]}:&& \text{ EPOC der gesamten Nachbelstung} \\[1ex]
&\text{EPOC}_{\text{ges}} \text{ [ml·kg}^{-1}\text{]}:&& \text{ Auf Körpermasse normierte EPOC-Werte} \\[1ex]
&\text{EPOC}_{\text{PCr}} \text{ [l]}:&& \text{ EPOC Menge die der Rephosphorylierung von PCr zugeordnet wird} \\[1ex]
&\text{EPOC}_{\text{PCr}} \text{ [ml·kg}^{-1}\text{]}: && \text{ Auf Körpermasse normierte EPOC der PCr-Rephosphorylierung} \\[1ex]
&\eta_{\text{Arbeit,sitzen}} \text{ [%]}: && \text{ Arbeitswirkungsgrad (sitzende Position)} \\[1ex]
&\eta_{\text{Brutto}} \text{ [%]}: && \text{ Bruttowirkungsgrad} \\[1ex]
&\eta_{\text{Delta}}: && \text{ Deltawirkungsgrad} \\[1ex]
&\eta_{\text{muskulär}} \text{ [%]}: && \text{ Muskulärer Wirkungsgrad} \\[1ex]
&\eta_{\text{Netto}} \text{ [%]}: && \text{ Nettowirkungsgrad} \\[1ex]
&\eta_{\text{Total}} \text{ [%]}: && \text{ Gesamtwirkungsgrad} \\[1ex]
&\text{kÄ} \text{ [kJ·l}^{-1}\text{]}:&& \text{ Kalorisches Äquivalent für einen bestimmten RQ-Wert} \\[1ex]
&\text{kÄ}_{\text{KH}} \text{ [kJ·l}^{-1}\text{]}:&& \text{ Kalorisches Äquivalent für Kohlenhydrate} \\[1ex]
&\text{nD [U·min}^{-1}\text{]}: && \text{ Drehzahl: Definiert als die Anzahl vollständiger Rotationen der Tretkurbel pro Zeiteinheit} \\[1ex]
&\text{O}_{2}\text{-Speicher} \text{ [l]}:&& \text{ Körpereigene Sauerstoffspeicher} \\[1ex]
&\text{P}_{\text{Int}} \text{ [W]}: && \text{ Innere Leistung} \\[1ex]
&\text{P}_{\text{mech}} \text{ [W]}: && \text{ Mechanische Leistung} \\[1ex]
&\text{P}_{\text{Tot}} \text{ [W]}: && \text{ Gesamtleistung} \\[1ex]
&\text{PCr}_{\text{Norm}} \text{ [mmol·kg}^{-1}\text{]}:&& \text{ Normwert der PCr-Konzentration} \\[1ex]
&\text{PCr}_{\text{used}} \text{ [mmol·kg}^{-1}\text{]}:&& \text{ Umgesetzte PCr-Menge pro kg Muskelfeuchmasse} \\[1ex]
&\text{P/O}_{2} \text{ [mol ATP·mol O}_2^{-1}\text{]}:&& \text{ P/O}_{2}\text{-Verhältnis} \\[1ex] 
&\dot{Q} \text{ [l·min}^{-1}\text{]}:&& \text{ Herzminutenvolumen} \\[1ex]
&R^2:&& \text{ Bestimmtheitsmaß der Modellanpassungen} \\[1ex]
&R_{2,\text{off}}: && \text{ Bestimmtheitsmaß der EPOC-Modellanpassungen} \\[1ex]
&\text{RQ}:&& \text{ Respiratorischer Quotient} \\[1ex]
&t_{\text{delay}} \text{ [s]}:&& \text{ Zeitverzögerung der EPOC-Modellanpassung} \\[1ex]
&t_{1/2} \text{ [s]}:&& \text{ Halbwertszeit} \\[1ex] 
&\tau_{\text{A}} \text{ [min]}:&& \text{ Zeitkonstante der schnellen EPOC-Komponente} \\[1ex]
&\tau_{\text{B}} \text{ [min]}:&& \text{ Zeitkonstante der langsamen EPOC-Komponente} \\[1ex]
&\tau_{\text{fast}} \text{ [min]}:&& \text{ Zeitkonstante der schnellen EPOC-Komponente} \\[1ex]
&\tau_{\text{on}} \text{ [s]}: && \text{ Zeitkonstante der monoexponentiellen Anpassung des Sauerstoffvolumenstroms} \\[1ex]
&\tau_{\text{slow}} \text{ [min]}:&& \text{ Zeitkonstante der langsamen EPOC-Komponente} \\[1ex]
&V_{\text{m,O}_2} \text{ [l·mol}^{-1}\text{]}:&& \text{ Molvolumen von Sauerstoff} \\[1ex]
&\text{VBV} \text{ [l]}:&& \text{ Venöses Blutvolumen} \\[1ex]
&\dot{\text{V}}\text{O}_{2} \text{ [l·min}^{-1}\text{]}:&& \text{ Sauerstoffvolumenstrom} \\[1ex]
&\dot{\text{V}}\text{O}_{2,\text{on,Start}} \text{ [l·min}^{-1}\text{]}: && \text{ V̇O}_2\text{ zu Beginn der Kinetik-Modellanpassung} \\[1ex]
&\dot{\text{V}}\text{O}_{2,\text{Referenz}} \text{ [l·min}^{-1}\text{]}:&& \text{ Sauerstoffvolumenstrom während der Referenzphase} \\[1ex]
&\dot{\text{V}}\text{O}_{2,\text{Ruhe}} \text{ [l·min}^{-1}\text{]}:&& \text{ Sauerstoffvolumenstrom in Ruhe} \\[1ex]
&\dot{\text{V}}\text{O}_{2,\text{Start}} \text{ [l·min}^{-1}\text{]}:&& \text{ Sauerstoffvolumenstrom zu Beginn der jeweiligen Belastungsintensität} \\[1ex]
&\text{W}_{\text{Aerob}} \text{ [kJ]}: && \text{ Aerobe Energiebereitstellung} \\[1ex]
&\text{W}_{\text{AER}} \text{ [kJ]}: && \text{ Aerobe Energiekomponente} \\[1ex]
&\text{W}_{\text{BLC}} \text{ [kJ]}: && \text{ Anaerob-laktazide Energiekomponente} \\[1ex]
&\text{W}_{\text{Ext}} \text{ [kJ]}: && \text{ Externe Arbeit} \\[1ex]
&\text{W}_{\text{Int}} \text{ [kJ]}: && \text{ Innere Arbeit} \\[1ex]
&\text{W}_{\text{PCr}} \text{ [kJ]}:&& \text{ Berechnete anaerobe-alaktazide Energiekomponente} \\[1ex]
&\text{W}_{\text{PCr,corrected}} \text{ [kJ]}:&& \text{ Berechnete anaerobe-alaktazide Energiekomponente abzüglich der Sauerstoffspeicher} \\[1ex]
&\text{W}_{\text{TOT}} \text{ [kJ]}: && \text{ Gesamtenergieumsatz aus allen Stoffwechselwegen} \\[1ex]
 \end{alignat*}
\]

\phantomsection\label{refs}
\begin{CSLReferences}{0}{1}
\end{CSLReferences}



\end{document}
