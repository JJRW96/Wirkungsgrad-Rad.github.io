% Options for packages loaded elsewhere
\PassOptionsToPackage{unicode}{hyperref}
\PassOptionsToPackage{hyphens}{url}
\PassOptionsToPackage{dvipsnames,svgnames,x11names}{xcolor}
%
\documentclass[
  letterpaper,
  DIV=11]{scrartcl}

\usepackage{amsmath,amssymb}
\usepackage{iftex}
\ifPDFTeX
  \usepackage[T1]{fontenc}
  \usepackage[utf8]{inputenc}
  \usepackage{textcomp} % provide euro and other symbols
\else % if luatex or xetex
  \usepackage{unicode-math}
  \defaultfontfeatures{Scale=MatchLowercase}
  \defaultfontfeatures[\rmfamily]{Ligatures=TeX,Scale=1}
\fi
\usepackage{lmodern}
\ifPDFTeX\else  
    % xetex/luatex font selection
\fi
% Use upquote if available, for straight quotes in verbatim environments
\IfFileExists{upquote.sty}{\usepackage{upquote}}{}
\IfFileExists{microtype.sty}{% use microtype if available
  \usepackage[]{microtype}
  \UseMicrotypeSet[protrusion]{basicmath} % disable protrusion for tt fonts
}{}
\makeatletter
\@ifundefined{KOMAClassName}{% if non-KOMA class
  \IfFileExists{parskip.sty}{%
    \usepackage{parskip}
  }{% else
    \setlength{\parindent}{0pt}
    \setlength{\parskip}{6pt plus 2pt minus 1pt}}
}{% if KOMA class
  \KOMAoptions{parskip=half}}
\makeatother
\usepackage{xcolor}
\setlength{\emergencystretch}{3em} % prevent overfull lines
\setcounter{secnumdepth}{-\maxdimen} % remove section numbering
% Make \paragraph and \subparagraph free-standing
\makeatletter
\ifx\paragraph\undefined\else
  \let\oldparagraph\paragraph
  \renewcommand{\paragraph}{
    \@ifstar
      \xxxParagraphStar
      \xxxParagraphNoStar
  }
  \newcommand{\xxxParagraphStar}[1]{\oldparagraph*{#1}\mbox{}}
  \newcommand{\xxxParagraphNoStar}[1]{\oldparagraph{#1}\mbox{}}
\fi
\ifx\subparagraph\undefined\else
  \let\oldsubparagraph\subparagraph
  \renewcommand{\subparagraph}{
    \@ifstar
      \xxxSubParagraphStar
      \xxxSubParagraphNoStar
  }
  \newcommand{\xxxSubParagraphStar}[1]{\oldsubparagraph*{#1}\mbox{}}
  \newcommand{\xxxSubParagraphNoStar}[1]{\oldsubparagraph{#1}\mbox{}}
\fi
\makeatother


\providecommand{\tightlist}{%
  \setlength{\itemsep}{0pt}\setlength{\parskip}{0pt}}\usepackage{longtable,booktabs,array}
\usepackage{calc} % for calculating minipage widths
% Correct order of tables after \paragraph or \subparagraph
\usepackage{etoolbox}
\makeatletter
\patchcmd\longtable{\par}{\if@noskipsec\mbox{}\fi\par}{}{}
\makeatother
% Allow footnotes in longtable head/foot
\IfFileExists{footnotehyper.sty}{\usepackage{footnotehyper}}{\usepackage{footnote}}
\makesavenoteenv{longtable}
\usepackage{graphicx}
\makeatletter
\def\maxwidth{\ifdim\Gin@nat@width>\linewidth\linewidth\else\Gin@nat@width\fi}
\def\maxheight{\ifdim\Gin@nat@height>\textheight\textheight\else\Gin@nat@height\fi}
\makeatother
% Scale images if necessary, so that they will not overflow the page
% margins by default, and it is still possible to overwrite the defaults
% using explicit options in \includegraphics[width, height, ...]{}
\setkeys{Gin}{width=\maxwidth,height=\maxheight,keepaspectratio}
% Set default figure placement to htbp
\makeatletter
\def\fps@figure{htbp}
\makeatother
% definitions for citeproc citations
\NewDocumentCommand\citeproctext{}{}
\NewDocumentCommand\citeproc{mm}{%
  \begingroup\def\citeproctext{#2}\cite{#1}\endgroup}
\makeatletter
 % allow citations to break across lines
 \let\@cite@ofmt\@firstofone
 % avoid brackets around text for \cite:
 \def\@biblabel#1{}
 \def\@cite#1#2{{#1\if@tempswa , #2\fi}}
\makeatother
\newlength{\cslhangindent}
\setlength{\cslhangindent}{1.5em}
\newlength{\csllabelwidth}
\setlength{\csllabelwidth}{3em}
\newenvironment{CSLReferences}[2] % #1 hanging-indent, #2 entry-spacing
 {\begin{list}{}{%
  \setlength{\itemindent}{0pt}
  \setlength{\leftmargin}{0pt}
  \setlength{\parsep}{0pt}
  % turn on hanging indent if param 1 is 1
  \ifodd #1
   \setlength{\leftmargin}{\cslhangindent}
   \setlength{\itemindent}{-1\cslhangindent}
  \fi
  % set entry spacing
  \setlength{\itemsep}{#2\baselineskip}}}
 {\end{list}}
\usepackage{calc}
\newcommand{\CSLBlock}[1]{\hfill\break\parbox[t]{\linewidth}{\strut\ignorespaces#1\strut}}
\newcommand{\CSLLeftMargin}[1]{\parbox[t]{\csllabelwidth}{\strut#1\strut}}
\newcommand{\CSLRightInline}[1]{\parbox[t]{\linewidth - \csllabelwidth}{\strut#1\strut}}
\newcommand{\CSLIndent}[1]{\hspace{\cslhangindent}#1}

\KOMAoption{captions}{tablesignature}
\makeatletter
\@ifpackageloaded{tcolorbox}{}{\usepackage[skins,breakable]{tcolorbox}}
\@ifpackageloaded{fontawesome5}{}{\usepackage{fontawesome5}}
\definecolor{quarto-callout-color}{HTML}{909090}
\definecolor{quarto-callout-note-color}{HTML}{0758E5}
\definecolor{quarto-callout-important-color}{HTML}{CC1914}
\definecolor{quarto-callout-warning-color}{HTML}{EB9113}
\definecolor{quarto-callout-tip-color}{HTML}{00A047}
\definecolor{quarto-callout-caution-color}{HTML}{FC5300}
\definecolor{quarto-callout-color-frame}{HTML}{acacac}
\definecolor{quarto-callout-note-color-frame}{HTML}{4582ec}
\definecolor{quarto-callout-important-color-frame}{HTML}{d9534f}
\definecolor{quarto-callout-warning-color-frame}{HTML}{f0ad4e}
\definecolor{quarto-callout-tip-color-frame}{HTML}{02b875}
\definecolor{quarto-callout-caution-color-frame}{HTML}{fd7e14}
\makeatother
\makeatletter
\@ifpackageloaded{caption}{}{\usepackage{caption}}
\AtBeginDocument{%
\ifdefined\contentsname
  \renewcommand*\contentsname{Inhaltsverzeichnis}
\else
  \newcommand\contentsname{Inhaltsverzeichnis}
\fi
\ifdefined\listfigurename
  \renewcommand*\listfigurename{Abbildungsverzeichnis}
\else
  \newcommand\listfigurename{Abbildungsverzeichnis}
\fi
\ifdefined\listtablename
  \renewcommand*\listtablename{Tabellenverzeichnis}
\else
  \newcommand\listtablename{Tabellenverzeichnis}
\fi
\ifdefined\figurename
  \renewcommand*\figurename{Abbildung}
\else
  \newcommand\figurename{Abbildung}
\fi
\ifdefined\tablename
  \renewcommand*\tablename{Tabelle}
\else
  \newcommand\tablename{Tabelle}
\fi
}
\@ifpackageloaded{float}{}{\usepackage{float}}
\floatstyle{ruled}
\@ifundefined{c@chapter}{\newfloat{codelisting}{h}{lop}}{\newfloat{codelisting}{h}{lop}[chapter]}
\floatname{codelisting}{Listing}
\newcommand*\listoflistings{\listof{codelisting}{Listingverzeichnis}}
\makeatother
\makeatletter
\makeatother
\makeatletter
\@ifpackageloaded{caption}{}{\usepackage{caption}}
\@ifpackageloaded{subcaption}{}{\usepackage{subcaption}}
\makeatother
\ifLuaTeX
\usepackage[bidi=basic]{babel}
\else
\usepackage[bidi=default]{babel}
\fi
\babelprovide[main,import]{ngerman}
% get rid of language-specific shorthands (see #6817):
\let\LanguageShortHands\languageshorthands
\def\languageshorthands#1{}
\ifLuaTeX
  \usepackage{selnolig}  % disable illegal ligatures
\fi
\usepackage{bookmark}

\IfFileExists{xurl.sty}{\usepackage{xurl}}{} % add URL line breaks if available
\urlstyle{same} % disable monospaced font for URLs
\hypersetup{
  pdftitle={Diskussion},
  pdflang={de},
  colorlinks=true,
  linkcolor={blue},
  filecolor={Maroon},
  citecolor={Blue},
  urlcolor={Blue},
  pdfcreator={LaTeX via pandoc}}

\title{Diskussion}
\author{}
\date{}

\begin{document}
\maketitle

In Stichpunkten: Aufbau der Diskussion anhand der formulierten
Forschungsfragen und Hypothesen:

\subsection{Forschungsfragen:}\label{forschungsfragen}

\begin{enumerate}
\def\labelenumi{\arabic{enumi}.}
\item
  \emph{Unterschiede zwischen den Körperpositionen Sitzen und Stehen?}
\item
  \emph{Unterschiede des η\textsubscript{muskulär} bei den
  Belastungsintensitäten?}
\item
  \emph{Unterschiede des η\textsubscript{muskulär} zwischen den
  Belastungsintensitäten in den Körperpositionen?}
\end{enumerate}

\subsection{Hypothesen:}\label{hypothesen}

\begin{enumerate}
\def\labelenumi{\arabic{enumi}.}
\item
  \emph{Keine signifikanten Unterschiede des η\textsubscript{muskulär}
  zwischen Sitzen und Stehen}
\item
  \emph{Keine signifikanten Unterschiede des η\textsubscript{muskulär}
  zwischen leichten, moderaten und schweren Belastungsintensitäten}
\item
  \emph{Unterschiede zwischen den Intensitäten der Bedingungen}

  \begin{enumerate}
  \def\labelenumii{\arabic{enumii}.}
  \item
    \emph{Bei niedrigen Intensitäten gibt es keine signifikanten
    Unterschiede des η\textsubscript{muskulär} zwischen sitzender und
    stehender Position.}
  \item
    \emph{Bei mittleren und hohen Intensitäten gibt es keine statistisch
    signifikanten Unterschiede des η\textsubscript{muskulär} zwischen
    sitzender und stehender Position}
  \end{enumerate}
\end{enumerate}

\begin{tcolorbox}[enhanced jigsaw, colback=white, titlerule=0mm, breakable, colbacktitle=quarto-callout-note-color!10!white, title=\textcolor{quarto-callout-note-color}{\faInfo}\hspace{0.5em}{Wirkungsgradberechnungen}, toptitle=1mm, colframe=quarto-callout-note-color-frame, bottomtitle=1mm, leftrule=.75mm, opacityback=0, arc=.35mm, left=2mm, coltitle=black, rightrule=.15mm, opacitybacktitle=0.6, toprule=.15mm, bottomrule=.15mm]

\begin{longtable}[]{@{}lc@{}}
\toprule\noalign{}
Wirkungsgrade & Berechnung \\
\midrule\noalign{}
\endhead
\bottomrule\noalign{}
\endlastfoot
η\textsubscript{Brutto} & P\textsubscript{mech} /
E\textsubscript{Aerob} \\
η\textsubscript{Netto} & P\textsubscript{mech} /
(E\textsubscript{Aerob}~-~E\textsubscript{Ruhe}) \\
η\textsubscript{Total} & P\textsubscript{mech} /
(E\textsubscript{Tot}~-~E\textsubscript{Ruhe}) \\
η\textsubscript{muskulär} & (P\textsubscript{mech} +
P\textsubscript{Int}) /
(E\textsubscript{Tot}~-~E\textsubscript{Ruhe}) \\
η\textsubscript{Arbeit} & P\textsubscript{mech} / (E\textsubscript{Tot}
-~E\textsubscript{Ruhe} -~E\textsubscript{Leerbewegung}) \\
η\textsubscript{delta} & ΔP\textsubscript{mech} /
ΔE\textsubscript{Tot} \\
\end{longtable}

\end{tcolorbox}

\subsubsection{Zu Hypothese 1}\label{zu-hypothese-1}

\begin{itemize}
\tightlist
\item
  η\textsubscript{muskulär} signifikant höher in sitzender Position
  (24.98 ± 1.06\% vs.~24.07 ± 0.96\%) (F(1,8) = 7.64, p = .024,
  η\textsubscript{p}\textsuperscript{2} = .489) \textbf{→ Hypothese
  widerlegt}

  \begin{itemize}
  \tightlist
  \item
    Effekt primär durch signifikant höhere P\textsubscript{Tot} in
    sitzender Bedingung bei konstanter W\textsubscript{TOT}
  \item
    Methodische Limitation: Berechnungsvalidität von
    P\textsubscript{Tot} bzw. P\textsubscript{Int} für Stehen-Bedingung
    ist zu hinterfragen
  \item
    Bei Ausklammerung von P\textsubscript{Int} kein signifikanter
    Unterschied → siehe η\textsubscript{Total}
  \end{itemize}
\item
  η\textsubscript{Netto} und η\textsubscript{Brutto} unterschieden sich
  nicht signifikant

  \begin{itemize}
  \tightlist
  \item
    Tendenziell höhere Werte in stehender Position
  \end{itemize}
\end{itemize}

\subsubsection{Zu Hypothese 2}\label{zu-hypothese-2}

\begin{itemize}
\item
  Signifikanter Effekt der Intensität auf η\textsubscript{muskulär}
  (F(2,16) = 5.23, p = .018, η\textsubscript{p}\textsuperscript{2} =
  .395). η\textsubscript{muskulär} nimmt mit steigender Intensität ab
  (24.94 ± 1.27\% - 24.14 ± 0.90\%) \textbf{→ Hypothese widerlegt}

  \begin{itemize}
  \tightlist
  \item
    Post-hoc-Analysen: Signifikanter Unterschied zwischen leicht und
    schwer (p = .074, d = .75)
  \item
    Moderate und schwere sowie moderate und leichte Intensitätsstufen
    ohne statistisch signifikanten Unterschied
  \end{itemize}
\item
  Signifikanter Effekt der Intensität auf η\textsubscript{muskulär}
  innerhalb der Sitzen Bedingung (F(2,16) = 10.19, p = .001,
  η\textsubscript{p}\textsuperscript{2} = .560). \textbf{→ Hypothese
  widerlegt}

  \begin{itemize}
  \tightlist
  \item
    Post-hoc-Analysen: Signifikanter Unterschied zwischen leicht und
    schwer (p = .017, d = 1.31).
  \item
    Moderat und leicht (p = .217, d = .83) sowie moderat und schwerer (p
    = .427, d = .63) nicht signifikant
  \end{itemize}
\item
  Wirkungsgradverhalten bei steigender Belastungsintensität:

  \begin{itemize}
  \item
    η\textsubscript{muskulär} und η\textsubscript{Total} sinken
    tendenziell ab, da W\textsubscript{TOT} überproportional ansteigt
    gegenüber P\textsubscript{TOT} für η\textsubscript{muskulär} bzw.
    P\textsubscript{mech} für η\textsubscript{Total}. Jedoch nur bei
    η\textsubscript{muskulär} signifikant, was womöglich damit zu
    erklären ist, dass P\textsubscript{Int} nicht ansteigt mit
    steigender Belastungsintensität und somit der
    η\textsubscript{muskulär} noch stärker abfällt als
    η\textsubscript{Total}
  \item
    η\textsubscript{Arbeit,sitzen} verhält sich mit steigender
    Belastungsintensität innerhalb der Sitzen Bedingung vergleichbar wie
    η\textsubscript{muskulär}, da zwischen der leichten und schweren
    Belastungsintensität jeweils ein signifikanter Unterschied besteht.
    Die Unterschiede bestehen nur darin, wie die Kosten für die innere
    Leistung bestimmt wurden. Für η\textsubscript{muskulär} wurden diese
    anhand des biomechanischen Modells oder 3D-Kinematik errechnet,
    während der Sauerstoffumsatz für die Bewegungen anhand des
    Drehzahltests in η\textsubscript{Arbeit,sitzen} berechnet wurde. So
    scheinen beide Berechnungsansätze valide Methoden für die Bestimmung
    der inneren Arbeit zu sein. Für die stehende Bedingung können die
    beiden nicht verglichen werden, da η\textsubscript{Arbeit} nicht für
    Stehen berechnet wurde
  \item
    η\textsubscript{Netto} zeigt gegensätzliches Verhalten bei
    steigender Belastungsintensität. Die Mittelwerte von
    η\textsubscript{Netto} steigen bei höheren Intensitäten im Mittel
    über beide Bedingungen, auch wenn keine signifikanten Unterschiede
    festzustellen sind. Innerhalb der jeweiligen Bedingungen zeigt sich
    ein sehr geringer nicht signifikanter Abfall für
    η\textsubscript{Netto}, bzw. ein fast konstanter Nettowirkungsgrad
    innerhalb der Sitzen Bedingung, aber während dem Stehen ein
    konstanter nicht signifikanter Anstieg von leicht zu schwer
  \item
    η\textsubscript{Brutto} steigt wie zu erwarten signifikant an mit
    steigender Intensität, was auf den kleiner werdenden prozentualen
    Anteil des Ruhenergieumsatzes zurückzuführen ist
  \end{itemize}
\end{itemize}

\subsubsection{Zu Hypothese 3}\label{zu-hypothese-3}

\begin{itemize}
\item
  Signifikanter Effekt zwischen der leichten Intensität zwischen Sitzen
  und Stehen von η\textsubscript{muskulär}. Sitzen signifikant höher
  (25.68 ± 1.03 vs.~24.20 ± 1.05). Die statistische Analyse zeigt eine
  hohe Signifikanz und Effektstärke (F(1,8) = 16.63, p = .004,
  η\textsubscript{p}\textsuperscript{2} = .675) → \textbf{Hypothese
  bestätigt}
\item
  Tendenziell signifikanter, aber nicht vollständig signifikanter Effekt
  zwischen der moderaten Intensität zwischen Sitzen und Stehen. Sitzen
  weiterhin numerisch höher (24.92 ± 0.78 vs.~24.08 ± 1.08) →
  \textbf{Hypothese tendentiell bestätigt}
\item
  Nicht signifikanter Effekt von η\textsubscript{muskulär} zwischen der
  schweren Intensität zwischen Sitzen und Stehen. Sitzen numerisch höher
  (24.36 ± 0.99 vs.~23.92 ± 0.81), jedoch zeigt statistische Prüfung
  keine relevante Differenz (F(1,8) = 1.41, p = .269,
  η\textsubscript{p}\textsuperscript{2} = .150) → \textbf{Hypothese
  bestätigt}
\item
  Unterschied in η\textsubscript{muskulär} zwischen Sitzen und Stehen
  nimmt mit steigender Intensität systematisch ab → \textbf{Hypothese
  bestätigt}
\item
  Alle anderen Wirkungsgradberechnungsmethoden zeigen keine
  vergleichbaren signifikanten Ergebnisse.

  \begin{itemize}
  \tightlist
  \item
    Die konsistent höheren P\textsubscript{Tot} bzw.
    P\textsubscript{Int}-Werte im Sitzen über alle
    Intensitätsbedingungen hinweg legen nahe, dass die beobachteten
    Unterschiede höchstwahrscheinlich auf eine erhöhte berechnete innere
    Arbeitsleistung zurückzuführen sind
  \end{itemize}
\end{itemize}

\subsubsection{\texorpdfstring{Erklärung der statistisch signifikanten
Unterschiede von
η\textsubscript{muskulär}}{Erklärung der statistisch signifikanten Unterschiede von ηmuskulär}}\label{erkluxe4rung-der-statistisch-signifikanten-unterschiede-von-ux3b7muskuluxe4r}

\begin{itemize}
\item
  η\textsubscript{muskulär} und η\textsubscript{Total} unterscheiden
  sich lediglich durch die Einbeziehung der inneren Arbeit. Die innere
  Arbeit variiert signifikant zwischen Sitzen (37.5 ± 11.3) und Stehen
  (22.1 ± 6.9).
\item
  Die P\textsubscript{Int} Mittelwerte bleiben über alle Intensitäten
  konstant, da sie drehzahlabhängig sind und die Drehzahl nahezu
  identisch bleibt. Dadurch zeigt sich eine deutlich größere Differenz
  zwischen η\textsubscript{muskulär} und η\textsubscript{Total} von
  durchschnittlich 0.80 bei leichter und 0.34 bei schwerer Intensität,
  weil P\textsubscript{Int} nahezu gleichbleibt, während der
  physiologische Energieumsatz steigt.
\item
  Da P\textsubscript{Int} im Sitzen signifikant höher war, ist
  η\textsubscript{muskulär} auch höher im Sitzen als im Stehen. Ohne
  Berücksichtigung von P\textsubscript{Int} (wie in
  η\textsubscript{Total}) zeigt sich keine Signifikanz. Daher
  repräsentiert η\textsubscript{Total} eher die in der Literatur
  erwarteten Verhaltensweisen.
\item
  SCHWÄCHEN VON P\textsubscript{int}:

  \begin{itemize}
  \tightlist
  \item
    Berechnung ohne Rücksicht der potentiellen ENergie bei der BEcnung
    der inneren Leistung. Deshalb wahrschinelich im sitzen unterschätz
    und im Stehen massiv untershcätzt. Aber für das stheen gibt es kien
    Verleichswerte
  \end{itemize}
\end{itemize}

\paragraph{\texorpdfstring{Fazit von
η\textsubscript{muskulär}:}{Fazit von ηmuskulär:}}\label{fazit-von-ux3b7muskuluxe4r}

\begin{itemize}
\item
  η\textsubscript{muskulär} wahrscheinlich nur für das Sitzen
  repräsentativ, aber für die stehenden Bedingung wahrscheinlich etwas
  unterschätzt. Hier also noch mehr Forschungsbedarf, um die
  P\textsubscript{Int} fürs Stehen noch präziser einschätzen zu können.

  \begin{itemize}
  \item
    η\textsubscript{muskulär} wahrscheinlich valide Ergebnisse für das
    Fahren im Sitzen, aber die Ergebnisse für das Stehen sind
    anzuzweifeln.
  \item
    Für den Vergleich der beiden Bedingungen wahrscheinlich
    η\textsubscript{Total} am besten geeignet und hier sind im Gegensatz
    zu η\textsubscript{muskulär} keine signifikanten Ergebnisse zu
    sehen.
  \end{itemize}
\end{itemize}

\subsection{Relevanz für die Praxis}\label{relevanz-fuxfcr-die-praxis}

\begin{itemize}
\item
  Das Fahren im Stehen stellt bezüglich des Wirkungsgrades eine valide
  Alternative zum Sitzen dar, um kurzzeitig die Körperposition zu
  variieren und unterschiedliche Muskelgruppen zu belasten. Wie
  beschrieben, ist der gemittelte η\textsubscript{muskulär} über alle
  Intensitäten in stehender Position niedriger als im Sitzen. Dieser
  Wirkungsgrad eignet sich jedoch, wie bereits diskutiert, für die
  Berechnungen im Sitzen nur bedingt als Bewertungsmethode.
\item
  Ein entscheidender Aspekt beim Radfahren in der Praxis, der in der
  Studie nicht berücksichtigt wurde, ist der signifikante Anstieg des
  Luftwiderstands beim Wechsel von sitzender zu stehender Position. Dies
  kann insbesondere bei Profiradsportlern, die hohe Geschwindigkeiten
  sowohl auf ebenen Strecken als auch in bergigen Terrains erreichen, zu
  einer substantiellen Reduktion der Geschwindigkeit bei
  gleichbleibender mechanischer Leistung führen. Daher ist das Fahren im
  Stehen in der Praxis primär für kurze Zeitintervalle an steilen
  Anstiegen oder in Situationen mit geringer Geschwindigkeit sinnvoll,
  bei denen der Luftwiderstand gegenüber der Gravitationskraft oder dem
  Rollwiderstand eine untergeordnete Rolle spielt. Eine potenzielle
  Ausnahme bilden Situationen, in denen kurzzeitig sehr hohe mechanische
  Leistungen erforderlich sind. Gemäß vorliegender Literatur kann im
  Stehen über kurze Zeiträume eine höhere mechanische Leistung generiert
  werden.
\end{itemize}

\subsection{Limitationen}\label{limitationen}

\begin{itemize}
\tightlist
\item
  Geringe Stichprobengröße
\item
  Leichte und moderate Intensität zu schwer
\item
  Innere Leistung im Stehen vermutlich unterschätzt, da für gleiche
  Drehzahl kaum höher als im Sitzen

  \begin{itemize}
  \tightlist
  \item
    Wurde vergleichbar berechnet wie im Sitzen
  \item
    Haltearbeit sowie vertikale Änderungen des Körperschwerpunktes nicht
    einbezogen
  \end{itemize}
\item
  Einfluss der leichten Belastung von 50 Watt in den Erholungsphasen auf
  die EPOC-Messung

  \begin{itemize}
  \tightlist
  \item
    Keine geregelte Trittrate in dieser Phase
  \end{itemize}
\item
  Zu kurze Erholung nach den jeweiligen Belastungen, vor allem an
  Testtag 2 nach dem Sprinttest

  \begin{itemize}
  \tightlist
  \item
    Sprinttest kontraproduktiv keine Rückkehr auf Laktat-Baseline
  \end{itemize}
\item
  Mögliche Ungenauigkeiten bei der Bestimmung der ventilatorischen
  Schwellen durch das 30s-Stufenprotokoll im Vergleich zum BDR-Protokoll
\item
  Mögliche Ungenauigkeiten bei der Modellierung der EPOC-Kurven und der
  Berechnung des WPCR aufgrund der 50-Watt Nachbelastung
\item
  Mögliche Ungenauigkeiten der Atemgasmessung bei hohen
  Ventilationsraten
\item
  Mögliche Ungenauigkeiten bei der Bestimmung der Körpersegmentmassen
  und -schwerpunkte für die 3D-Bewegungsanalyse
\item
  \ldots{}
\end{itemize}

\subsection{Ausblick}\label{ausblick}

\begin{itemize}
\tightlist
\item
  Weitere Studien mit größerer Stichprobengröße, um die Validität der
  Ergebnisse zu erhöhen, gibt teilweise große Effekte, die aber nicht
  signifikant sind
\item
  Vergleich der verwendeten EPOC-Berechnungsmethode mit anderen Ansätzen
  aus der Literatur
\item
  Berücksichtigung der Wärmeabgabe bei der Berechnung des
  Gesamtenergieumsatzes und des Wirkungsgrades, bzw. Berechnung des
  Wirkungsgrades über die Wärmeabgabe
\item
  Einfluss der Torque Efficiency oder Pedal-Smoothness (hohe
  Leistungsspitzen) auf den Wirkungsgrad
\item
  Einfluss von Kurbellängen auf den Wirkungsgrad
\item
  \ldots{}
\item
  Zusammenhang zwischen negativer meachnischer Leistung und innerer
  Arbeit erforschen
\item
  Sind Liestungsspitzen shcneller ermüdend als gleichmäßige Leistungne
  -\textgreater{} Pedal Smoothness
\end{itemize}

\section{Quellenverzeichnis}\label{quellenverzeichnis}

\phantomsection\label{refs}
\begin{CSLReferences}{0}{1}
\end{CSLReferences}



\end{document}
